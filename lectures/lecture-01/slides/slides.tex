\documentclass{beamer}
\usetheme{Madrid}

\usepackage{pgf,pgfarrows,pgfnodes,pgfautomata,pgfheaps,pgfshade}
\usepackage{amsmath,amssymb}
\usepackage[utf8]{inputenc}
\usepackage{colortbl}
\usepackage[english]{babel}
\usepackage{booktabs}
\usepackage{slpython}

\author{Luís Pedro Coelho}
\institute{Programming for Scientists}

\graphicspath{{figures/}{figures/generated/}{images/}}

\AtBeginSection[] % Do nothing for \subsection*
{
	\begin{frame}<beamer>
		\frametitle{Outline}
		\tableofcontents[currentsection,currentsubsection]
	\end{frame}
}


\title{First Lecture: Course Policies and Course Overview}

\begin{document}

\section{Basics}

\frame{\frametitle{Introduction}

\begin{block}{Who am I}
\begin{itemize}
\item Luís Pedro Coelho (lpc@cmu.edu)
\item Third year Ph.D.\ student in computational biology
\item My office is 409D, Mellon Institute
\end{itemize}
\end{block}

\begin{block}{This course}
\begin{itemize}
\item Meets: Tuesdays \& Thursdays 6.30
\item Tuesday: Lecture
\item Thursday: Lab Session
\item Office Hours:  (or by appointment)
\item Course Website: http://coupland.cbi.cmu.edu/~luispedro/PfS-2008
\item Course Mailing List:
\end{itemize}
\end{block}

}
\frame{\frametitle{Homework}

\begin{block}{Homeworks}
\begin{itemize}
\item Grades will come from homeworks and final project.
\item Homeworks will be assigned on Tuesdays and are due the next Tuesday at mid-night.
\item Up to 24 hours delay: 20\% penalty. Up until beginning of lab-session: 30\% penalty.
\item Thursday sessions will often include discussion of homeworks, so no after that, you can no longer turn in homeworks.
\item Homework will normally consists of a multiple-choice/short answer section plus a programming question.
\item Homeworks are to be turned in by email in text (.txt) or pdf format for the questions, and Python code for the programming (.py files).\\
    In particular, \textbf{doc, docx, odt} will not be accepted.
\end{itemize}
\end{block}
}
\frame{\frametitle{Project}

\begin{block}{Project}
\begin{itemize}
\item There will be a final project, which will replace homeworks towards the end of the class.
\item You can submit your own project, but I will have my own proposals.
\item You can work individually or in small groups. Expectations will scale linearly.
\end{itemize}
\end{block}
}

\frame{\frametitle{Course Structure}
\begin{block}{Two Lectures}
\begin{enumerate}
\item Tuesday session: a lecture where basic concepts are presented.
\item Tuesday session: a lab type session, where we discuss particular technologies that let us implement the basic concepts. You can use a laptop in this lecture, but I recommend pen \& paper.
\end{enumerate}
\end{block}
}
\frame{\frametitle{}
}
\frame{\frametitle{}
}
\frame{\frametitle{}
}
\section{Overview}
\frame{\frametitle{Programming for Scientists}

Programming
for
Scientists

}

\frame{\frametitle{}
}
\frame{\frametitle{Course Overview}
\begin{block}{Class Modules}
\begin{enumerate}
\item Structured Programming
\item Basics of Scientific Programming
\item Advanced/Applied Topics
\end{enumerate}
\end{block}
}
\frame{\frametitle{Course Overview: Module I}

\begin{block}{Structure of Programs}
\begin{itemize}
\item Structured programming and general good programming
\item Object-based programming
\item Object-oriented programming
\end{itemize}
\end{block}
}
\frame{\frametitle{Basics of Scientific Programming}

\begin{block}{Scientific Programming}
\begin{enumerate}
\item Representation of numbers, memory usage
\item Numerical optimisation
\item Aspects of stochastic programming
\item Distribution of software
\end{enumerate}
\end{block}
}
\frame{\frametitle{Advanced Topics}

This is a mixed bag and subject to change (if you want to).

\begin{block}{Advanced Topics}
\begin{itemize}
\item Graphical user interfaces
\item Concurrent programming
\item Databases
\item Interfacing multiple languages (lab session only)
\end{itemize}
\end{block}
}
\section{Class Starts Here}

\begin{frame}[fragile]
\frametitle{Structured Programming}

\begin{block}{Bad Code}
\begin{python}
def xyf(x,y):
    t=x*0.5
    t2=t/2
    t3=x*p
    t4=t2*t2-y
    t4=sqrt(t4)
    t5=t2-t4,t2+t4
    return t5

\end{python}
\end{block}

\begin{block}{Good Code}
\begin{python}
def pq\_formula(p,q):
    return -p/2.+sqrt(p*p/4.-q), -p/2.-sqrt(p*p/4.-q)

\end{python}
\end{block}
\end{frame}

\frame{\frametitle{Principles of Good Code}

\begin{block}{Principle and Rules}
\begin{itemize}
\item A \alert{principle} is a general goal.
\item A \alert{rule} is a concrete way to implement principles.\\
    Often a rule is only a ``rule of thumb.''
\end{itemize}
\end{block}

\begin{block}{Principles of Good Code}
\begin{itemize}
\item Correct
\item Sufficiently efficient
\item Readable
\item Extendable
\end{itemize}
\end{block}
}

\begin{frame}[fragile]
\frametitle{Homework 0}

\begin{block}{Homework (for next week)}
Run the following Python program and email me the results:
\begin{python}
email="YOUR EMAIL"
print email, hash(email)
\end{python}
\end{block}
\end{frame}

\end{document}
