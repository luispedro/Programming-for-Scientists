\documentclass{beamer}
\usetheme{Madrid}

\usepackage{pgf,pgfarrows,pgfnodes,pgfautomata,pgfheaps,pgfshade}
\usepackage{amsmath,amssymb}
\usepackage[utf8]{inputenc}
\usepackage{colortbl}
\usepackage[english]{babel}
\usepackage{booktabs}
\usepackage{slpython}

\author{Luís Pedro Coelho}
\institute{Programming for Scientists}

\graphicspath{{figures/}{figures/generated/}{images/}}

\AtBeginSection[] % Do nothing for \subsection*
{
	\begin{frame}<beamer>
		\frametitle{Outline}
		\tableofcontents[currentsection,currentsubsection]
	\end{frame}
}


\title{Project Case Study}
\begin{document}
\frame{\maketitle}

\note{First, let's get some stuff out of the way\ldots}

\begin{frame}[fragile]
\frametitle{First Steps for the Project}
\begin{enumerate}
\item Email me your preferences
\item Get a google account
\item Send me the email for the google account (just email)
\item Wait for my confirmation
\item Checkout the repository (see google page)
\item Add yourself to AUTHORS.txt
\item Commit
\end{enumerate}
\end{frame}

\begin{frame}[fragile]
\frametitle{Project Structure}
\begin{itemize}
\item COPYING.MIT
\item AUTHORS.txt
\item README
\item template.py
\item setup.py
\item docs/
\item tests/
\item particles/
\end{itemize}
\end{frame}

\begin{frame}[fragile]
\frametitle{reStructuredText}

\begin{verbatim}
----------------
This is a title
----------------

Now here comes my text. *This is bold*

Here's a list
    * first element
    * second element
\end{verbatim}

\end{frame}

\begin{frame}[fragile]
\begin{python}
def brownian_motion(mv_s,p_dis,crowding,crowding_std):
    '''
    particles = brownian_motion(mv_s,p_dis,crowding,crowding_std):

    Generate particle tracks according to Brownian motion.

    Parameters
    -----------

        * mov_sigma: std. dev. for movement
                (i.e., the position of a particle at time T+1
                 is given by a Gaussian centred on its position at time T
                 with std. dev. mov_sigma)
        * p_disappear: probability that a particle disappears
        * crowding: expected number of particles
        * crowding_std: std. dev. of number of particles

    The initial set number of particles is sampled from a
    normal distribution of mean crowding and std. dev.
    crowding_std. New particles are randomly generated to 
    offset the dying particles and maintain the property
    that the expected number of particles is given by
    crowding and crowding_std.
    '''
\end{python}

\end{frame}

\begin{frame}[fragile]
\frametitle{Software \& Software Product}

A product is only finished when you have the \alert{software}, \alert{documentation}, \alert{tests}, and a \alert{web-page}.
\end{frame}

\note{Now, let's get to the project.}

\begin{frame}[fragile]
\frametitle{Project Goals}

\begin{enumerate}
\item Generate a video of particles moving
\begin{enumerate}[a]
\item<2-> Generate tracks
\begin{enumerate}[i]
\item<3-> \alert<4>{Brownian motion} (with parameters)
\item<3-> Brownian motion w. momentum
\item<3-> \ldots
\end{enumerate}
\item<2-> Generate video from tracks
\begin{enumerate}[i]
\item<3-> \alert<4>{Shot noise} (parameters)
\item<3-> Blur
\item<3-> \ldots
\end{enumerate}
\end{enumerate}
\item Track the particles in this movie
\begin{enumerate}[a]
\item<2-> Detect particles
\begin{enumerate}[i]
\item<3-> \alert<4>{Global threshold}
\item<3-> Local threshold
\item<3-> \ldots
\end{enumerate}
\item<2-> Track particles
\end{enumerate}
\item Compare
\item Visualise
\end{enumerate}
\end{frame}

\begin{frame}[fragile]
\frametitle{Initial Guess}

\begin{python}
tracks = generate_tracks()
video = generate_video(tracks)
recovered = track(video)
statistics = compare(tracks,recovered)
print_statistics(statistics)
\end{python}
\end{frame}

\end{document}
