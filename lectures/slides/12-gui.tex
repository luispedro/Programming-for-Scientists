\documentclass{beamer}
\usetheme{Madrid}

\usepackage{pgf,pgfarrows,pgfnodes,pgfautomata,pgfheaps,pgfshade}
\usepackage{amsmath,amssymb}
\usepackage[utf8]{inputenc}
\usepackage{colortbl}
\usepackage[english]{babel}
\usepackage{booktabs}
\usepackage{slpython}

\author{Luís Pedro Coelho}
\institute{Programming for Scientists}

\graphicspath{{figures/}{figures/generated/}{images/}}

\AtBeginSection[] % Do nothing for \subsection*
{
	\begin{frame}<beamer>
		\frametitle{Outline}
		\tableofcontents[currentsection,currentsubsection]
	\end{frame}
}


\title{GUI Programming I}
\begin{document}
\frame{\maketitle}

\begin{frame}[fragile]
\frametitle{How to Write a GUI}
Many options:
\begin{itemize}
\item tk
\item wxWindows
\item Gtk
\item \alert<2>{Qt}
\item \ldots
\end{itemize}
\end{frame}

\begin{frame}[fragile]
\frametitle{Main Elements}
\begin{itemize}
\item Widgets
\item Event loop
\item Signals \& Slots
\end{itemize}
\end{frame}

\begin{frame}[fragile]
\frametitle{What's a widget?}
Everything you see on your screen is a widget.
\end{frame}


\begin{frame}[fragile]
\frametitle{Event Loop}
\begin{block}{Traditional Program}
\begin{python}
tracks = generate_tracks()
video = generate_video(tracks)
recovered = track(video)
statistics = compare(tracks,recovered)
print_statistics(statistics)
\end{python}
\end{block}

\begin{block}{Event Loop}
\begin{python}
while True:
    event = get_next_event()
    handle_event(event)
\end{python}
\end{block}
\end{frame}

\begin{frame}[fragile]
\frametitle{Hollywood Principle}
Don't call us, we'll call you.
\end{frame}

\begin{frame}[fragile]
\frametitle{Qt}
\begin{itemize}
\item Commercial software from Trolltech (now owned by Nokia).
\item Used to be GPL, now LGPL.
\item Very complete.\note{well documented}
\item C++ toolkit with Python bindings.
\end{itemize}
\end{frame}

\begin{frame}[fragile]
\frametitle{Actually Writing An Application}
\begin{enumerate}
\item Code it from scratch
\item Use a dialog editor
\end{enumerate}
\end{frame}

\begin{frame}[fragile]
\frametitle{Simple Example}

\begin{python}
import sys
from PyQt4 import QtGui
app = QtGui.QApplication(sys.argv)

widget = QtGui.QLabel('Hello World')
widget.setWindowTitle('Hello')
widget.show()

sys.exit(app.exec_())

\end{python}

\end{frame}

\begin{frame}[fragile]
\frametitle{Variation}


\begin{python}
import sys
from PyQt4 import QtGui
app = QtGui.QApplication(sys.argv)

button = QtGui.QPushButton('Press Me')
button.setWindowTitle('Hello')
button.show()

sys.exit(app.exec_())

\end{python}

\end{frame}

\begin{frame}[fragile]
\frametitle{Signals \& Slots}

\begin{python}
bigger = True
def resize():
    global bigger
    w = button.width()
    h = button.height()
    if bigger:
        button.resize(w*2, h*2)
    else:
        button.resize(w//2, h//2)
    bigger = not bigger

\end{python}
\end{frame}

\begin{frame}[fragile]
\frametitle{Signals \& Slots (II)}

\begin{python}
import sys
from PyQt4 import QtGui, QtCore
app = QtGui.QApplication(sys.argv)

button = QtGui.QPushButton('Press Me')
button.setWindowTitle('Hello')
button.show()

app.connect(button, QtCore.SIGNAL('clicked()'), resize)
sys.exit(app.exec_())

\end{python}
\end{frame}

\begin{frame}[fragile]
\frametitle{Some Builtin Widgets}
\begin{columns}
\column[t]{.5\textwidth}
\begin{itemize}
\item QLabel
\item QPushButton
\item QRadioButton
\item QCheckBox
\item QLineEdit
\end{itemize}
\column[t]{.5\textwidth}
\begin{itemize}
\item QComboBox
\item QListView
\item QGroupBox
\item \ldots
\end{itemize}
\end{columns}
\end{frame}

\begin{frame}[fragile]
\frametitle{Object Oriented Widgets}
\begin{block}{Widget Properties \& Methods}
\begin{itemize}
\item size \& resize()
\item show() \& hide()
\item \ldots
\end{itemize}
\end{block}
\end{frame}
\end{document}
