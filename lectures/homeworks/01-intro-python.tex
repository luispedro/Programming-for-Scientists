\documentclass[article,twoside]{memoir}
\usepackage[latin9]{inputenc}
\usepackage{psfrag}
\usepackage{hyperref}
\usepackage{eepic}
\usepackage{color}
\usepackage{amsmath}
\usepackage{amssymb}
\usepackage{bm}
\usepackage{amsfonts}
\usepackage{amsthm}
\usepackage{sublabel}
\usepackage[vlined,linesnumbered,ruled]{algorithm2e}
\usepackage{multirow}
%\usepackage[]{pgf}
\usepackage[]{graphicx}

\graphicspath{{../figures/other/}{../figures/generated/}}


\DeclareMathOperator*{\argmax}{arg\,max}
\DeclareMathOperator*{\argmin}{arg\,min}

\newcommand{\pd}[2]{\frac{\partial#1}{\partial#2}}
\newcommand{\pdd}[2]{\frac{\partial^2 #1}{\partial #2^2}} 
\newcommand{\pdpd}[3]{\frac{\partial^2 #1}{\partial #2 \partial #3}} 
\newcommand{\B}{\ensuremath{\mathbb{B}}}
\newcommand*{\Pa}{\ensuremath{\text{\upshape\textbf{Pa}}}}
\newcommand*{\Dkl}{\ensuremath{D_{\text{\textsc{kl}}}}}
\newcommand*{\encspace}{\quad}

\newtheorem{theorem}{Theorem}
\newtheorem{lemma}{Lemma}
\newtheorem{definition}{Definition}

\newcommand*{\slantfrac}[2]{\hbox{$\raisebox{-.4ex}{$\,^#1$}\!/_#2$}}
\newcommand*{\onehalf}{\ensuremath{
{\frac{1}{2}}%
}}
\newcommand*{\dx}{\,dx}
\newcommand*{\dt}{\,dt}

\newcommand*{\Expected}{\ensuremath{\mathbb{E}}}
\newcommand*{\Var}{\ensuremath{\text{Var}}}
\newcommand*{\Reals}{\ensuremath{\mathbb{R}}}
\newcommand*{\Binary}{\ensuremath{\mathbb{B}}}
\newcommand*{\discr}{\mbox{discr}}
\newcommand*{\from}{\leftarrow}

\newcommand*{\indicator}[1]{\hspace{1pt}[\hspace{-.4em}[\hspace{3pt} #1 \hspace{3pt}]\hspace{-.4em}]\hspace{2pt}}
\newcommand*{\Assign}{\ensuremath\,:=\,}
\newcommand*{\vect}[1]{\ensuremath{\bm{#1}}}
\newcommand*{\textvalue}[1]{\mbox{\textsl{#1}}}
\newcommand*{\bigO}{\mathcal{O}}
\newcommand*{\MI}{\mbox{MI}}

\DeclareMathOperator{\goodness}{score}
\DeclareMathOperator{\powerset}{Pow}

\title{Student Survey}
\author{Programming for Scientists}

\bibliographystyle{alpha}
%\bibliographystyle{plain}
\pagestyle{Ruled}

\aliaspagestyle{chapter}{Ruled}
\makeatletter
\if@twoside
	\makeoddhead{Ruled}{}{}{\scshape\rightmark}
	\makeevenhead{Ruled}{\scshape\leftmark}{}{}
	\makeevenfoot{Ruled}{\pagenumberfont\thepage}{}{}
	\makeoddfoot{Ruled}{}{}{\pagenumberfont\thepage}
	% Put section number on top
	\def\sectionmark#1{\markright{#1 (\thesection)}}
	\def\chaptermark#1{\markboth{#1}{#1}}
	\renewcommand*{\bibmark}{\markboth{\bibname}{\bibname}} % I don't like empty headings!
\else

	% Put section number on top
	\def\sectionmark#1{\markright{#1 (\thesection)}}
	\def\chaptermark#1{\markright{#1 (\thechapter)}}

	\makeoddhead{Ruled}{}{}{\scshape\rightmark}
	\makeevenhead{Ruled}{}{}{\scshape\rightmark}
	\makeevenfoot{Ruled}{}{}{\pagenumberfont\thepage}
	\makeoddfoot{Ruled}{}{}{\pagenumberfont\thepage}
	\renewcommand*{\bibmark}{\markright{\bibname}} % I don't like empty headings!


\fi
\makeatother

%Change fonts: Page number sans-serif (much cleaner than the roman version)
\def\pagenumberfont{\sffamily}
% Change fonts: Section headers Sans-Serif:
\setsecheadstyle{\Large\sffamily\raggedright}
\setsubsecheadstyle{\large\sffamily\raggedright}
\setsubsubsecheadstyle{\normalsize\sffamily\raggedright}

% Title
\pretitle{\LARGE\sffamily}
\posttitle{\par\vspace{4ex}}
\preauthor{\large\sffamily\hspace{1cm}}
\postauthor{\par\vspace{3ex}}
\predate{\small\sffamily\hspace{1cm}Last Updated on: }
\postdate{\par\vspace{2cm}}
\copypagestyle{title}{plain}
\makeoddfoot{title}{}{}{}
\makeevenfoot{title}{}{}{}

\renewcommand{\abstractnamefont}{\sffamily}
\renewcommand{\abstracttextfont}{}
\renewcommand{\absnamepos}{flushleft}


\makechapterstyle{mestrado}{% Originally ``AlexanderGrebenkov'', adapted
\renewcommand{\chapterheadstart}{\goodbreak\vspace*{\beforechapskip}\medskip}
\renewcommand{\chapnamefont}{\normalfont\Large\scshape}
\renewcommand{\chapnumfont}{\normalfont\Large\scshape}
\renewcommand{\chaptitlefont}{\normalfont\Large\scshape}
\renewcommand{\printchaptername}{}
\renewcommand{\chapternamenum}{}
\renewcommand{\printchapternum}{\normalfont\Large\scshape\S\thechapter}
\renewcommand{\afterchapternum}{\hspace{1em}}
\renewcommand{\afterchaptertitle}{\par\nobreak\vspace{-.9em}\moveright 6pt\vbox to 1pt{\hrule width .64\textwidth}\nobreak\vskip\afterchapskip\nobreak}
}
\chapterstyle{mestrado}

\SetCommentSty{textsl}

\newcommand*{\question}{\textbf{Question: }}

\begin{document}

\author{Programming for Scientists}
\title{Homework 1}
\date{Jan 27}
\begin{document}
\maketitle

\chapter{Questions}

\question
Which function returns the number of elements in a list?

\question
How do you access the first element of a list? How do you access the last?

\question
Python has two forms of adding non-code text. You use the \#~sign to start a \textit{comment} (which extends to the end of the line). Or, as we saw in class, you can use a string at the start of the function to add a \textit{documentation string}.

\begin{python}
def function(arg):
    '''
    This is a documentation string.
    '''
    print arg # this is a comment
\end{python}

What is, then, the difference between a comment and documentation? Why these two mechanisms?

\question
Alice and Bob were doing some \textit{pair programming}\footnote{Pair programming is when two programmers write code together. Normally, one will sit at the computer typing with the other will look over his shoulder to make sure he isn't making any mistakes. After a while, they switch.

You might consider pair programming to help with assignments in this class.} and Alice wrote the following code:

\begin{python}
amount = 100
interest = 5
amount *= (1. + interest)

print 'Value after interest:', amount 
\end{python}

Bob argued that this would never work. ``In Python, numbers are immutable. In the line \lstinline{amount *= (1. + interest)}, you are changing the value of \lstinline{amount}. However, when they try it out, it works as Alice expected. Where is the mistake in Bob's reasoning? Are numbers mutable after all?

\question
What is the difference between the following two code examples:

A)

\begin{python}
A = [1, 2, 3]
B = [1, 2, 3]
\end{python}

B)

\begin{python}
A = [1, 2, 3]
B = A
\end{python}

Write a small piece of code (should be 2 or 3 lines) that behaves differently if you insert it after each of the two segments above.

\chapter{Programming Assignment}

You are considering whether to enroll in a couple of different saving plans. You put some money in now and deduct some initial fee, you put in some money every month, interest accrues each month and is added to the pot (at the end of the month, just before you add in more). How much money would you have after 5 years under each plan?

\begin{tabular}{lcccc}
\toprule
Plan Name & Fee & Initial Amount & Monthly Amount & Interest \\
\midrule
Low-fee & 0 & 0 & 100 & .2\% \\
Low-interest & 20 & 100 & 50 & .02\% \\
High-initial & 100 & 1000 & 50 & .2\% \\
High-monthly & 100 & 0 & 250 & .2\% \\
High-interest & 100 & 0 & 100 & .4\% \\
\bottomrule
\end{tabular}

\end{document}

