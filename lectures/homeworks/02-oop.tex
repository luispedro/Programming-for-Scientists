\documentclass[article,twoside]{memoir}
\usepackage[latin9]{inputenc}
\usepackage{psfrag}
\usepackage{hyperref}
\usepackage{eepic}
\usepackage{color}
\usepackage{amsmath}
\usepackage{amssymb}
\usepackage{bm}
\usepackage{amsfonts}
\usepackage{amsthm}
\usepackage{sublabel}
\usepackage[vlined,linesnumbered,ruled]{algorithm2e}
\usepackage{multirow}
%\usepackage[]{pgf}
\usepackage[]{graphicx}

\graphicspath{{../figures/other/}{../figures/generated/}}


\DeclareMathOperator*{\argmax}{arg\,max}
\DeclareMathOperator*{\argmin}{arg\,min}

\newcommand{\pd}[2]{\frac{\partial#1}{\partial#2}}
\newcommand{\pdd}[2]{\frac{\partial^2 #1}{\partial #2^2}} 
\newcommand{\pdpd}[3]{\frac{\partial^2 #1}{\partial #2 \partial #3}} 
\newcommand{\B}{\ensuremath{\mathbb{B}}}
\newcommand*{\Pa}{\ensuremath{\text{\upshape\textbf{Pa}}}}
\newcommand*{\Dkl}{\ensuremath{D_{\text{\textsc{kl}}}}}
\newcommand*{\encspace}{\quad}

\newtheorem{theorem}{Theorem}
\newtheorem{lemma}{Lemma}
\newtheorem{definition}{Definition}

\newcommand*{\slantfrac}[2]{\hbox{$\raisebox{-.4ex}{$\,^#1$}\!/_#2$}}
\newcommand*{\onehalf}{\ensuremath{
{\frac{1}{2}}%
}}
\newcommand*{\dx}{\,dx}
\newcommand*{\dt}{\,dt}

\newcommand*{\Expected}{\ensuremath{\mathbb{E}}}
\newcommand*{\Var}{\ensuremath{\text{Var}}}
\newcommand*{\Reals}{\ensuremath{\mathbb{R}}}
\newcommand*{\Binary}{\ensuremath{\mathbb{B}}}
\newcommand*{\discr}{\mbox{discr}}
\newcommand*{\from}{\leftarrow}

\newcommand*{\indicator}[1]{\hspace{1pt}[\hspace{-.4em}[\hspace{3pt} #1 \hspace{3pt}]\hspace{-.4em}]\hspace{2pt}}
\newcommand*{\Assign}{\ensuremath\,:=\,}
\newcommand*{\vect}[1]{\ensuremath{\bm{#1}}}
\newcommand*{\textvalue}[1]{\mbox{\textsl{#1}}}
\newcommand*{\bigO}{\mathcal{O}}
\newcommand*{\MI}{\mbox{MI}}

\DeclareMathOperator{\goodness}{score}
\DeclareMathOperator{\powerset}{Pow}

\title{Student Survey}
\author{Programming for Scientists}

\bibliographystyle{alpha}
%\bibliographystyle{plain}
\pagestyle{Ruled}

\aliaspagestyle{chapter}{Ruled}
\makeatletter
\if@twoside
	\makeoddhead{Ruled}{}{}{\scshape\rightmark}
	\makeevenhead{Ruled}{\scshape\leftmark}{}{}
	\makeevenfoot{Ruled}{\pagenumberfont\thepage}{}{}
	\makeoddfoot{Ruled}{}{}{\pagenumberfont\thepage}
	% Put section number on top
	\def\sectionmark#1{\markright{#1 (\thesection)}}
	\def\chaptermark#1{\markboth{#1}{#1}}
	\renewcommand*{\bibmark}{\markboth{\bibname}{\bibname}} % I don't like empty headings!
\else

	% Put section number on top
	\def\sectionmark#1{\markright{#1 (\thesection)}}
	\def\chaptermark#1{\markright{#1 (\thechapter)}}

	\makeoddhead{Ruled}{}{}{\scshape\rightmark}
	\makeevenhead{Ruled}{}{}{\scshape\rightmark}
	\makeevenfoot{Ruled}{}{}{\pagenumberfont\thepage}
	\makeoddfoot{Ruled}{}{}{\pagenumberfont\thepage}
	\renewcommand*{\bibmark}{\markright{\bibname}} % I don't like empty headings!


\fi
\makeatother

%Change fonts: Page number sans-serif (much cleaner than the roman version)
\def\pagenumberfont{\sffamily}
% Change fonts: Section headers Sans-Serif:
\setsecheadstyle{\Large\sffamily\raggedright}
\setsubsecheadstyle{\large\sffamily\raggedright}
\setsubsubsecheadstyle{\normalsize\sffamily\raggedright}

% Title
\pretitle{\LARGE\sffamily}
\posttitle{\par\vspace{4ex}}
\preauthor{\large\sffamily\hspace{1cm}}
\postauthor{\par\vspace{3ex}}
\predate{\small\sffamily\hspace{1cm}Last Updated on: }
\postdate{\par\vspace{2cm}}
\copypagestyle{title}{plain}
\makeoddfoot{title}{}{}{}
\makeevenfoot{title}{}{}{}

\renewcommand{\abstractnamefont}{\sffamily}
\renewcommand{\abstracttextfont}{}
\renewcommand{\absnamepos}{flushleft}


\makechapterstyle{mestrado}{% Originally ``AlexanderGrebenkov'', adapted
\renewcommand{\chapterheadstart}{\goodbreak\vspace*{\beforechapskip}\medskip}
\renewcommand{\chapnamefont}{\normalfont\Large\scshape}
\renewcommand{\chapnumfont}{\normalfont\Large\scshape}
\renewcommand{\chaptitlefont}{\normalfont\Large\scshape}
\renewcommand{\printchaptername}{}
\renewcommand{\chapternamenum}{}
\renewcommand{\printchapternum}{\normalfont\Large\scshape\S\thechapter}
\renewcommand{\afterchapternum}{\hspace{1em}}
\renewcommand{\afterchaptertitle}{\par\nobreak\vspace{-.9em}\moveright 6pt\vbox to 1pt{\hrule width .64\textwidth}\nobreak\vskip\afterchapskip\nobreak}
}
\chapterstyle{mestrado}

\SetCommentSty{textsl}

\newcommand*{\question}{\textbf{Question: }}

\begin{document}

\author{Programming for Scientists}
\title{Homework 2}
\date{Feb 4}
\begin{document}
\maketitle

Since this homework was posted a day late, you get an extra day.

\chapter{Questions}

\question
What is the \lstinline{__init__} method for? When is it called?

\question
In this class, we defined \textit{polymorphism} as the ability to write code that \emph{works with more than one type}. In Python, if you want code that works with two type of ``Bacteria'', that is achieved through:

\begin{enumerate}[a]
\item Inheritance: you define a base interface Bacterium and derive your working types from that.
\item Duck-typing: your types should all have methods with the same names.
\item Duck-typing: your derived type should have a name like the name of the base class (e.g., EnhancedBacterium deriving from Bacterium).
\item None of the above.
\end{enumerate}

\question
Consider the following code:

\begin{python}
class Point2(object):
    def __init__(self,x,y):
        self.x = x
        self.y = y

    def dist2(self):
        return self.x**2 + self.y**2 

p = Point2(2,2)
print p.dist2()
p.y = 0
print p.dist2()
print p.x
print p.y
\end{python}

This code prints four numbers. What are they:

\begin{enumerate}[a]
\item 8, 8, 2, 0
\item 8, 8, 2, 2
\item 8, 4, 2, 0
\item 8, 4, 2, 2
\end{enumerate}

\question
What is wrong with the following code?

\begin{python}
mynumbers = [0,1,2,3]
dictionary = {}
dictionary[mynumbers] = 'Lottery Winning Key'
\end{python}

\question
What is the difference between a \lstinline{set} and a \lstinline{frozenset}?

\chapter{Programming Assignment}

Recall the competing bacteria environment from class. Consider the following line of reasoning:

\begin{quote}
Species that mutate slowly do better when the environment is fixed, but worse when the environment is changing. Therefore, the best would be species that senses whether it is adapted to the environment and changes its reproduction mechanism to have higher mutation rates when the environment is more different from the adaptation.
\end{quote}

Your assignment is to 

\begin{enumerate}
\item Formulate a reasonable mathematical formulation of the idea above.
\item Implement it in Python. For this, download the file \textit{bacteria.py} from the website (this is the code used in lecture). You \textbf{should not} change the code in this file! You should simply import it and implement a new type of bacteria. The file \textit{simulatebac.py} is an example of the use of bacteria.py. You use it as a starting point or start your own script.
\end{enumerate} 

\end{document}
