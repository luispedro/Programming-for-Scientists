\documentclass{beamer}
\usetheme{Madrid}

\usepackage{pgf,pgfarrows,pgfnodes,pgfautomata,pgfheaps,pgfshade}
\usepackage{amsmath,amssymb}
\usepackage[utf8]{inputenc}
\usepackage{colortbl}
\usepackage[english]{babel}
\usepackage{booktabs}
\usepackage{slpython}

\author{Luís Pedro Coelho}
\institute{Programming for Scientists}

\graphicspath{{figures/}{figures/generated/}{images/}}

\AtBeginSection[] % Do nothing for \subsection*
{
	\begin{frame}<beamer>
		\frametitle{Outline}
		\tableofcontents[currentsection,currentsubsection]
	\end{frame}
}


\title{Numpy}
\begin{document}
\frame{\maketitle}

\begin{frame}[fragile]

\frametitle{Historical}
\note{
    Without numpy, this class would have to use something other than Python!
}

\begin{itemize}
\item Numeric (1995)
\item Numarray (for large arrays)
\item scipy.core (briefly, around 2005)
\item numpy (2005)
\end{itemize}
\end{frame}

\begin{frame}[fragile]
\frametitle{Currently}
\begin{itemize}
\item numpy 1.6
\item \alert{de facto} standard\note{ Not part of standard library, but no one uses anything else.}
\item very stable
\end{itemize}
\end{frame}

\begin{frame}[fragile]
\frametitle{Basic Type}

\lstinline{numpy.array} or \lstinline{numpy.ndarray}.

\bigskip
Multi-dimensional array of numbers.

\end{frame}

\begin{frame}[fragile]
\frametitle{numpy example}

\begin{python}
import numpy as np
A = np.array([
    [0,1,2],
    [2,3,4],
    [4,5,6],
    [6,7,8]])
print A[0,0]
print A[0,1]
print A[1,0]
\end{python}

\note{
    comment on the np abbreviation.
}
\end{frame}

\begin{frame}[fragile]
\frametitle{Some Array Properties}

\begin{python}
import numpy as np
A = np.array([
    [0,1,2],
    [2,3,4],
    [4,5,6],
    [6,7,8]])
print A.shape
print A.size
\end{python}
\end{frame}

\begin{frame}[fragile]
\frametitle{Some Array Functions}
\begin{python}
...
print A.max()
print A.min()
\end{python}

\begin{itemize}
\item max(): maximum
\item min(): minimum
\item ptp(): spread (max - min)
\item sum(): sum
\item std(): standard deviation
\item \ldots
\end{itemize}

\end{frame}

\begin{frame}[fragile]
\frametitle{Other Functions}
\begin{itemize}
\item \lstinline{np.exp}
\item \lstinline{np.sin}
\item \ldots
\end{itemize}

All of these work \alert{element-wise}!
\end{frame}

\begin{frame}[fragile]
\frametitle{Arithmetic Operations}
\begin{python}
import numpy as np
A = np.array([0,1,2,3])
B = np.array([1,1,2,2])

print A + B
print A * B
print A / B
\end{python}

\note{Operations are element-wise!}
\end{frame}

\begin{frame}[fragile]
\frametitle{Broadcasting}
Mixing arrays of different dimensions

\begin{python}
import numpy as np
A = np.array([
    [0,0,1],
    [1,1,2],
    [1,2,2],
    [3,2,2]
    ])
B = np.array([2,1,2])

print A + B
print A * B
\end{python}
\note{
    Present this as array of arrays instead of 2D-array.
}
\end{frame}

\begin{frame}[fragile]
\frametitle{Broadcasting}
Special case: scalar.

\begin{python}
import numpy as np
A = np.arange(100)
print A + 2
A += 2
\end{python}
\end{frame}

\begin{frame}[fragile]
\frametitle{Data Types}

\lstinline{numpy.ndarray} is a homogeneous array of numbers.

\begin{block}{Types}
\begin{itemize}
\item Boolean
\item int8, int16, \ldots
\item uint8, uint16,\ldots
\item float32, float64,\ldots
\item \ldots
\end{itemize}
\end{block}

Some types are only available in some platforms (e.g., \lstinline{float96}).

\end{frame}

\begin{frame}[fragile]
\frametitle{Object Construction}

\begin{python}
import numpy as np
A = np.array([0,1,1],np.float32)
A = np.array([0,1,1],float)
A = np.array([0,1,1],bool)
\end{python}
\end{frame}

\begin{frame}[fragile]
\frametitle{Reduction}

\begin{python}
A = np.array([
    [0,0,1],
    [1,2,3],
    [2,4,2],
    [1,0,1]])
print A.max(0)
print A.max(1)
print A.max()
\end{python}
prints
\begin{verbatim}
[2,4,3]
[1,3,4,1]
4
\end{verbatim}

The same is true for many other functions.
\end{frame}

\begin{frame}[fragile]
\frametitle{Slicing}

\begin{python}
import numpy as np
A = np.array([
    [0,1,2],
    [2,3,4],
    [4,5,6],
    [6,7,8]])
print A[0]
print A[0].shape
print A[1]
print A[:,2]
\end{python}
\end{frame}

\begin{frame}[fragile]
\frametitle{Slices Share Memory!}
\begin{python}
import numpy as np
A = np.array([
    [0,1,2],
    [2,3,4],
    [4,5,6],
    [6,7,8]])
B = A[0]
B[0] = -1
print A[0,0]
\end{python}

\end{frame}

\begin{frame}[fragile]
\frametitle{Pass is By Reference}

\begin{python}
def double(A):
    A *= 2

A = np.arange(20)
double(A)
\end{python}
\pause
\begin{python}
A = np.arange(20)
B = A.copy()
\end{python}
\end{frame}

\begin{frame}[fragile]
\frametitle{Logical Arrays}
\begin{python}
A = np.array([-1,0,1,2,-2,3,4,-2])
print (A > 0)
\end{python}
\end{frame}
\begin{frame}[fragile]
\frametitle{Logical Arrays II}
\begin{python}
A = np.array([-1,0,1,2,-2,3,4,-2])
print ( (A > 0) & (A < 3) ).mean()
\end{python}

What does this do?
\note{Prints the fraction of elements between 0 and 3.}
\end{frame}

\begin{frame}[fragile]
\frametitle{Logical Indexing}
\begin{python}
A[A < 0] = 0
\end{python}
or
\begin{python}
A *= (A > 0)
\end{python}
\end{frame}


\begin{frame}[fragile]
\frametitle{Logical Indexing}

\begin{python}
print 'Mean of positives', A[A > 0].mean()
\end{python}
\end{frame}

\begin{frame}[fragile]
\frametitle{Some Helper Functions}

\begin{block}{Constructing Arrays}
\begin{python}
A = np.zeros((10,10), dtype=np.int8)
B = np.ones(10)
C = np.arange(100).reshape((10,10))
...
\end{python}
\end{block}

\begin{block}{Multiple Dimensions}
\begin{python}
img = np.zeros((1024,1024,3),dype=np.uint8)
\end{python}
\end{block}

\end{frame}


\begin{frame}[fragile]
\frametitle{Documentation}

\href{http://docs.scipy.org/doc/}{http://docs.scipy.org/doc/}

\end{frame}


\end{document}
