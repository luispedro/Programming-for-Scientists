\documentclass{beamer}
\usetheme{Madrid}

\usepackage{pgf,pgfarrows,pgfnodes,pgfautomata,pgfheaps,pgfshade}
\usepackage{amsmath,amssymb}
\usepackage[utf8]{inputenc}
\usepackage{colortbl}
\usepackage[english]{babel}
\usepackage{booktabs}
\usepackage{slpython}

\author{Luís Pedro Coelho}
\institute{Programming for Scientists}

\graphicspath{{figures/}{figures/generated/}{images/}}

\AtBeginSection[] % Do nothing for \subsection*
{
	\begin{frame}<beamer>
		\frametitle{Outline}
		\tableofcontents[currentsection,currentsubsection]
	\end{frame}
}


\title{Distribution of Software}
\begin{document}
\frame{\maketitle}

\begin{frame}[fragile]
\frametitle{Why Make Your Software Available?}

\begin{itemize}
\item You \alert{want} to.
\item You \alert{have} to.
\end{itemize}
\note{Mention ``reproduceable research'' movement.
Mention BLAST.
}%
\end{frame}

\begin{frame}[fragile]
\frametitle{Science Code Manifesto}

\url{http://sciencecodemanifesto.org/}

\pause

Many other similar iniciatives.
\note{
    Probably some areas (sub-areas) of science will move towards this, others will not.

    Discuss science reproducibility aspect.
    Discuss moral aspects. Why some people feel this should be mandatory or do it themselves.
}%
\end{frame}


\begin{frame}[fragile]
\frametitle{Open Source}
What is open-source?
\end{frame}

\begin{frame}[fragile]
\frametitle{Intellectual Property}
Wikipedia-level introduction to intellectual property:
\begin{enumerate}
\item copyright
\item patent \note{software is patentable in the US, but not in other jurisdictions [EU]}
\item trademark
\end{enumerate}
\end{frame}

\begin{frame}[fragile]
\frametitle{Licenses}
A license is a \alert{loosening} of the default ``All Rights Reserved'' framework.
\note{
    Note that the code is owned either by he who writes it or by their lab/university/\ldots
}
\end{frame}

\begin{frame}[fragile]
\frametitle{Licenses}
\begin{enumerate}
\item Public-domain
\item BSD or MIT licenses
\item GPL (copy-left)
\item \ldots (10~million others)
\end{enumerate}
\end{frame}

\begin{frame}[fragile]
\frametitle{Public Domain}

I can do whatever I want.

\note{You can release the code into the public domain in some jurisdictions.}
\end{frame}

\begin{frame}[fragile]
\frametitle{MIT License}

Permission is hereby granted, free of charge, to any person
obtaining a copy of this software and associated documentation
files (the "Software"), to deal in the Software without
restriction, \alert{including without limitation the rights to use,
copy, modify, merge, publish, distribute, sublicense, and/or sell
copies of the Software}, and to permit persons to whom the
Software is furnished to do so, subject to the following
conditions:

The above copyright notice and this permission notice shall be
included in all copies or substantial portions of the Software.
\end{frame}

\begin{frame}[fragile]
\frametitle{BSD 3-Clause license}

Redistribution and use in source and binary forms, with or without
modification, are permitted provided that the following conditions are met:
\begin{enumerate}
\item Redistributions of source code must retain the above copyright notice, this list of conditions and the following disclaimer.
\item Redistributions in binary form must reproduce the above copyright notice, this list of conditions and the following disclaimer in the documentation and/or other materials provided with the distribution.
\item Neither the name of the <organization> nor the names of its contributors may be used to endorse or promote products derived from this software without specific prior written permission.
\end{enumerate}

\end{frame}

\begin{frame}[fragile]
\frametitle{Copyleft}

\centering
\includegraphics[width=.8\textwidth]{RMS.jpeg}

\end{frame}

\begin{frame}[fragile]
\frametitle{GNU Project}

\centering
\includegraphics[height=.6\textheight]{gnu-head.png}

\centering
\href{http://www.gnu.org/}{http://www.gnu.org}

\note{
    The GNU project is run by the free software foundation (also
    founded by rms).
}%
\end{frame}


\begin{frame}[fragile]
\frametitle{Free Software}

\begin{itemize}
\item Free as in beer (gratis)
\item Free as in speech (libre)
\end{itemize}

\note{Free software is a more ideological term than open source.

Which is why this lecture is termed ``open source''!
}%
\end{frame}

\begin{frame}[fragile]
\frametitle{GNU General Public License (GPL)}

You can distribute \& modify the software as long as
\begin{itemize}
\item you make the source code available,
\item you license your modifications under the GPL.
\end{itemize}

\note{
    There are several versions of the GPL. Currently in use
        * v2 \& v2.1
        * v3: June 2007.
}
\end{frame}

\begin{frame}[fragile]
\frametitle{GPL Compatibility}

You can use BSD licensed code in a GPL.\\
BSD is \alert{GPL compatible}.

\end{frame}

\begin{frame}[fragile]
\frametitle{GNU Lesser General Public License (LGPL)}

If you write a library and release it under the LGPL,\\
then closed-source programs can use it.

\end{frame}

\begin{frame}[fragile]
\frametitle{GPL Logo}

\centering
\includegraphics[width=.7\textwidth]{GPLlogo.png}

\end{frame}

\begin{frame}[fragile]
\frametitle{Summary}

\begin{itemize}
\item Think about your license
\item Pick either BSD/MIT, or GPL, or LGPL
\item Be careful about code re-use
\end{itemize}
\end{frame}

\begin{frame}[fragile]
\frametitle{Non-Code Open Source}

\begin{itemize}
\item Creative Commons.
\item GFDL
\item \ldots
\end{itemize}
\note{
    GFDL is the GNU Free Documentation License, on which Wikipedia is based.
}
\end{frame}

\begin{frame}[fragile]
\frametitle{Creative Commons}

\begin{block}{Restrictions}
\begin{itemize}
\item \alert{Attribution} [By] you need to state the source
\item \alert{Non-Commercial} [NC] you cannot use it for commercial purposes
\item \alert{Non-Derivative} [ND] you may not modify the work
\item \alert{Share-Alike} [SA] you must license your derivative works similarly
\end{itemize}
\note{
    These are all a bit 'undefined' and mostly un-tested by case law.

    What counts as non-commercial, for example?
}
\end{block}

Mix \& match to get a license, e.g., By-NC-SA.\note{
    This is the license of the class.
}
\end{frame}

\begin{frame}[fragile]
\frametitle{This is Complicated\ldots}
\ldots{} can't I just put up my software on my lab webpage?

\pause

Let me give you two answers:

\pause

\begin{enumerate}
\item Yes\only<3>{, but that is a confusing legal grey area and some people
(like me) will complain.}

\item No\only<3>{, because doing so is the same as writing ``this is made
available for you to look at, but I might change my mind at any minute and
demand that you start paying, take it offline, \&c.'' Because, the default is
``all rights reserved.''}
\end{enumerate}

\end{frame}

\begin{frame}[fragile]
{}
How do you actually distribute your software?

\begin{itemize}
\item Put it on your home page
\item Put it in PyPI (Python Package Index)
\item Use a service like \alert{github}, \alert{google code}, \alert{launchpad}, \ldots
\end{itemize}
\end{frame}

\begin{frame}[fragile]
\frametitle{Distributing Software: PyPI}

Python Package Index

\end{frame}

\end{document}
