\documentclass{beamer}
\usetheme{Madrid}

\usepackage{pgf,pgfarrows,pgfnodes,pgfautomata,pgfheaps,pgfshade}
\usepackage{amsmath,amssymb}
\usepackage[utf8]{inputenc}
\usepackage{colortbl}
\usepackage[english]{babel}
\usepackage{booktabs}
\usepackage{slpython}

\author{Luís Pedro Coelho}
\institute{Programming for Scientists}

\graphicspath{{figures/}{figures/generated/}{images/}}

\AtBeginSection[] % Do nothing for \subsection*
{
	\begin{frame}<beamer>
		\frametitle{Outline}
		\tableofcontents[currentsection,currentsubsection]
	\end{frame}
}


\title{Introduction to Python Programming}
\begin{document}

\frame{\maketitle}

\note{
    Goals for this session:

    - Be able to run a program.
    - Be able to run interactively.

    - Rough understanding of name/object concept.
    - Understand control flow.
}

\begin{frame}[fragile]
\frametitle{Python}

Let's digress for a moment discussing the language\ldots
\end{frame}


\begin{frame}[fragile]
\frametitle{Python Language History}

\begin{block}{History}
Python was started in the late 80's. It was intended to be both \alert{easy to teach} and \alert{industrial strength}.

It is (has always been) open-source and has become one of the most widely used languages (top 10).
\end{block}
\end{frame}

\begin{frame}
\frametitle{Python Versions}

\begin{block}{Python Versions}
\begin{itemize}
\item The current version of Python is \alert{2.7} and \alert{3.3}
\item This class assumes you have 2.6--2.7
\item There are some small differences when compared to version 3.x
\end{itemize}
\end{block}

\end{frame}

\begin{frame}[fragile]
\frametitle{Python Example}

\begin{python}
print "Hello World"
\end{python}
\end{frame}


\begin{frame}[fragile]

\begin{block}{Running Python}
\begin{enumerate}
\item From a file
\item Interactively
\end{enumerate}
\end{block}

\end{frame}

\begin{frame}[fragile]
\frametitle{Back to the Python Language}
Let's look at the language itself.
\end{frame}

\begin{frame}[fragile]
\frametitle{What is a Computer?}

\begin{enumerate}
\item Memory
\item Processor
\item Magic
\end{enumerate}
\end{frame}

\begin{frame}[fragile]
\frametitle{Python Model}

\begin{enumerate}
\item Objects
\item Operations on objects
\item Magic
\end{enumerate}
\end{frame}

\begin{frame}[fragile]
\frametitle{Computer Program}

\begin{block}{helloword.py}
\begin{python}
print 'Hello World'
\end{python}
\end{block}
\end{frame}

\begin{frame}[fragile]
\frametitle{Running a Program}
\begin{enumerate}
\item Shell
\item IDE
\end{enumerate}
\end{frame}

\begin{frame}[fragile]

\bigskip
\bigskip
\bigskip
Let me show you a demonstration\ldots

\note{
\begin{enumerate}
\item Demo shell with vim
\item Demo one IDE
\end{enumerate}
}


\end{frame}

\begin{frame}[fragile]
\frametitle{More Complex Example}

What is 25 + 20\%?

\pause
\begin{python}
print 25 * 1.2
\end{python}
\note{Demo the python shell.

This is basically a glorified calcular}
\end{frame}

\begin{frame}[fragile]
\frametitle{Blackboard demonstration}

\note{Use the blackboard to introduce the idea of objects, values and names.}
\end{frame}

\begin{frame}[fragile]
\frametitle{Conditionals}

\begin{python}

if <condition>:
    <statement 1>
    <statement 2>
else:
    <statement 3>

\end{python}
\end{frame}

\begin{frame}[fragile]
\frametitle{Lists}

\begin{python}

students = ['Luis','Mark','Rita']

print students[0]
print students[1]
print students[2]
\end{python}

\end{frame}

\begin{frame}[fragile]
\frametitle{Loops}

\begin{python}
students = ['Luis','Mark','Rita',...]

for st in students:
    print st
\end{python}
\end{frame}

\begin{frame}[fragile]
\frametitle{Loops (II)}

\begin{python}

students = ['Luis',...]

looking_for = 'Anna'

i = 0
while students[i] != looking_for:
    i += 1

if students[i] == looking_for:
    print 'Found her'
else:
    print "Sorry, she's not registered!"
\end{python}

\end{frame}

\begin{frame}
\frametitle{For Monday}

\begin{itemize}
\item Install Python(x,y)\\
    (or the equivalent on your platform)
\end{itemize}
\end{frame}

\end{document}
